%!TEX TS-program = xelatex
\documentclass[a4paper,12pt]{article}
\usepackage[OT6, T1]{fontenc}
\usepackage[utf-8]{inputenc}
\usepackage[english, russian]{babel}%% загружает пакет многоязыковой вёрстки

\def\changemargin#1#2{\list{}{\rightmargin#2\leftmargin#1}\item[]}
\let\endchangemargin=\endlist 

\usepackage{fontspec}
\setmainfont[
BoldFont=brillb.ttf,
ItalicFont=brilli.ttf,
BoldItalicFont=brillbi.ttf
]{brill.ttf}
	
	%%% Дополнительная работа с математикой
    \usepackage{float}
	\usepackage{amsmath,amsfonts,amssymb,amsthm,mathtools} % AMS
	\usepackage{icomma} % "Умная" запятая: $0,2$ --- число, $0, 2$ --- перечисление
	
	%%% Работа с картинками
	\usepackage{wrapfig} % Обтекание рисунков текстом
	\usepackage{subcaption}
	\usepackage{rotating}
	\usepackage{hhline}
	\usepackage{lscape}
	\usepackage[usenames,dvipsnames,svgnames,table,rgb]{xcolor}%пакет для использования цветов

	\usepackage[inline]{enumitem}
	
	%%% Работа с таблицами
	\usepackage{array,tabularx,tabulary,booktabs} % Дополнительная работа с таблицами
	\usepackage{longtable} % Длинные таблицы
	\usepackage{multirow} % Слияние строк в таблице
	
	\usepackage{multicol} % Несколько колонок
	
	%%% Страница
	\usepackage{extsizes} % Возможность сделать 14-й шрифт
	\usepackage{geometry} % Простой способ задавать поля
	\geometry{top=30mm}
	\geometry{bottom=30mm}
	\geometry{left=30mm}
	\geometry{right=30mm}
	
	%\usepackage{fancyhdr} % Колонтитулы
	% \pagestyle{fancy}
	%\renewcommand{\headrulewidth}{0pt} % Толщина линейки, отчеркивающей верхний колонтитул
	% \lfoot{Нижний левый}
	% \rfoot{Нижний правый}
	% \rhead{Верхний правый}
	% \chead{Верхний в центре}
	% \lhead{Верхний левый}
	% \cfoot{Нижний в центре} % По умолчанию здесь номер страницы
	
	\usepackage{setspace} % Интерлиньяж
	\onehalfspacing % Интерлиньяж 1.5
	%\doublespacing % Интерлиньяж 2
	%\singlespacing % Интерлиньяж 1
	
	\usepackage{lastpage} % Узнать, сколько всего страниц в документе.
	\usepackage{soul} % Модификаторы начертания
	\usepackage{bbding}
	\definecolor{dark-gray}{gray}{0.3}
	\usepackage{hyperref}
	\usepackage[usenames,dvipsnames,svgnames,table,rgb]{xcolor}
	\hypersetup{ % Гиперссылки
	colorlinks=true, % false: ссылки в рамках; true: цветные ссылки
	linkcolor=dark-gray, % внутренние ссылки
	citecolor=black, % на библиографию
	filecolor=black, % на файлы
	urlcolor=ForestGreen % на URL
	}
	
	\usepackage{environ}
	\makeatletter
	\newsavebox{\measure@tikzpicture}
	\NewEnviron{scaletikzpicturetowidth}[1]{%
	\def\tikz@width{#1}%
	\def\tikzscale{1}\begin{lrbox}{\measure@tikzpicture}%
	\BODY
	\end{lrbox}%
	\pgfmathparse{#1/\wd\measure@tikzpicture}%
	\edef\tikzscale{\pgfmathresult}%
	\BODY
	}
	\makeatother
	
    \usepackage{verbatim}
	
	\usepackage{attachfile2}
	\attachfilesetup{appearance=true,
	color=0 0 0
	}
	
	%%% Лингвистические пакеты
	%\usepackage{savetrees} % пакет, который экономит место
	\usepackage{forest} % для рисования деревьев
	\usepackage{vowel} % для рисования трапеций гласных
	\usepackage{natbib}
	\bibliographystyle{ugost2008ns.bst}
	\bibpunct[: ]{[}{]}{;}{a}{}{,}

	\usepackage{philex} % пакет для примеров
    
	\renewcommand{\thesection}{\arabic{section}.}
	\renewcommand{\thesubsection}{\arabic{section}.\arabic{subsection}}
	\setlength{\columnsep}{1.6cm}
	
	\usepackage{sectsty}
	\sectionfont{\normalsize}
	\subsectionfont{\normalsize}
	\usepackage{titlesec}
	\titlespacing*{\section}
	{0pt}{2ex plus 0ex minus .2ex}{0ex plus .2ex}
	\titlespacing*{\subsection}
	{0pt}{2ex plus 0ex minus .2ex}{0ex plus .2ex}
	\newlength{\bibitemsep}\setlength{\bibitemsep}{.2\baselineskip plus .05\baselineskip minus .05\baselineskip}
	\newlength{\bibparskip}\setlength{\bibparskip}{0pt}
	\let\oldthebibliography\thebibliography
	\renewcommand\thebibliography[1]{%
	\oldthebibliography{#1}%
	\setlength{\parskip}{\bibitemsep}%
	\setlength{\itemsep}{\bibparskip}%
	}
\usepackage{tikz}
\usetikzlibrary{arrows,positioning,shapes.geometric}
\usepackage{alltt}
\usepackage{bibunits}
\usepackage{enumitem}
\usepackage{ltablex,booktabs}
\begin{document}
\definecolor{zzttqq}{rgb}{0.26666666666666666,0.26666666666666666,0.26666666666666666}
\definecolor{cqcqcq}{rgb}{0.7529411764705882,0.7529411764705882,0.7529411764705882}
\thispagestyle{empty}
\begin{center}
\noindent 

\textbf{NATIONAL RESEARCH UNIVERSITY HIGHER SCHOOL OF ECONOMICS}\\
\textbf{Faculty of Humanities}\\
\textbf{School of Linguistics}\\
\vfill

\huge{PROJECT PROPOSAL}\\
\large
\LARGE
\textbf{Profiling of the locative participant as the basis of lexical oppositions}\\
\Large
Профилирование локативного участника как основа лексических противопоставлений\\
\vfill
\vfill
\normalsize
\begin{flushright}
Matvey Sokolovskiy, 153\bigskip\\
Linguistic Supervisor:\\
Yury Lander, Ph. D., \\
Associate Professor, School of Linguistics\\
                       
Scientific Supervisor:\\
Tatiana Reznikova, Ph. D., \\
Associate Professor, School of Linguistics\\

\end{flushright}
\vfill
\begin{center}
Moscow --- 2019
\end{center}

\end{center}
\pagebreak

\begin{changemargin}{2.5cm}{0.5cm}
\textit{Abstract}\\

%The argument structure of verbs appears to vary in different situations, includingthe same locative participant (LocPar)

The locative participant (LocPar) of the situation can be marked differently. In some cases the LocPar is in the position of the direct object (e.g. \textit{to visit X}), in other cases it can be expressed in PP (e.g. \textit{to put something into X}). The key object of this research is to find and explore the derivation, which allows to place the LocPar to the position of the direct object, observed in Russian, and cases of lexical opposition based on the position of this participant (e.g. \textit{to steal VS to rob}). As a result of the typological overview there is going to be a list of frames profiling the LocPar diversely in different languages, and that will give the opportunity to predict the argument structure of a verb, describing a situation with LocPar on the basis of its semantics.
\end{changemargin}
\normalsize



%{
%  \hypersetup{linkcolor=black}
%  \renewcommand{\contentsname}{Table of contents}
%  \tableofcontents
%}

\section{Introduction} 


\section{Main part}
\subsection{Literature review}
The present research 

Since the project is going to have a typological orientation, the good way to start is reviewing the methods applicable to a typological research. Those can be found in this book by Martin Haspelmath \citep{haspelmath2001language}. The comparison of languages' lexicalisations shifts us to the 'younger' field of lexical typology, wich takes roots in the famous article of Brent Berlin and Paul Kay about the typology of color naming \citep{berlin1991basic} and later described by Christian Lehmann in \citep{lehmann1990towards} and Maria Koptjevskaja-Tamm in \citep{koptjevskaja2008approaching} anf \citep{krv2015}. But the main focus of the research is argument structure, so the project exists in between lexical and grammatical typologies.

The problem of verb's argument structure and, in particular, of Russian verb's argument structure has been studied deeply. During the development of the conceptual model "Cмысл $\Leftrightarrow$ Tекст" (known as the Meaning-Text theory, invented and developed by Igor Mel'{\v{c}}uk and a number of Soviet linguists), the problem was analysed a lot, and its mentions can be found in \citep[119-156]{апресян1995избранные}, \citep[129-131]{апресян1995избранные2} by Yuri Apresjan and in \citep[134–139]{мельчук1974опыт} by Igor Mel’{\v{c}}uk. A number of more narrow cases, directly related to the present study was discussed among Russian linguists later in \citep{муравенко1998случаях}, \citep{цинман1998модель}. Haspelmath also described the problem from morphological point of view in \citep{haspelmath2004valency}.

After all, the discussion of the lexical opposition \textit{to rob -- to steal}, which is considered as the starting point of the current research, appears in \citep[45-48]{goldberg1995constructions}, \citep{thorgren2005transaction} and \citep{van2007role}. The semantic field of 'embezzlement' in Russian contains much more lexicalisations and it is discussed at different angles by Yuri Muratov in his PH.D. thesis \citep{муратовгенетическая} and some earlier articles. But there is no analysis of the field from the argument structure and valency's point of view. 


\subsection{Methods}
The major resources to explore verbs' argument structures are various text corpora and dictionaries' examples. So, the idea is to start with constructing the list of Russian ditransitive (or tritransitive) verbs, having the LocPar among it's arguments. All these verbs can be conditionally\footnote{The division is called conditional because, for example, there is a  Russian verb \textit{загружать (to load)}. It has two different government patterns: object \textit{(what is loaded)} in accusative case + LocPar \textit{(where is loaded)} in PP and vice versa. It cannot be reffered to the first group, because the default government pattern is considered to be the first one. And it cannot be reffered to the second group either, because there is no derrivation. So we would place it between these two given groups.} divided into three groups:
\begin{enumerate}
    \item Verbs, having the LocPar as the direct object originally.
    \item Verbs, which can be derived, thus the LocPar is profiled in the direct object.
    \item Verbs, which do not allow profiling of the LocPar.
\end{enumerate}

After getting the list it is possible to find those verbs in other languages with help of parallel corpora provided by Russian National Corpus (\small{ \url{http://ruscorpora.ru/}}) \normalsize  and bilingual dictionaries. Parallel corpora give us the opportunity to explore, how the same situation's lexical expression is arranged in several languages. Combining the information about verbs from all discussed languages we will compile the list of semantic frames\footnote{Basic polyvalent sittuations including the LocPar.}. Then it can be sorted by the position of the LocPar, which means from frames, wich often have a basic lexeme with LocPar in the position of direct object to those, wich cannot express the LocPar in the position of direct object at all (or almost never can). 

\subsection{Anticipated results}

It is considered, that the position of direct object is prototypically occupied by the patient. However the insances of placing there the LocPar instead of the patient are pretty common. We hope to find the correlation between the syntactical position of the LocPar and its role in the situation generally. 

\section{Conclusion}

%The locative participant of a situation was found 

\section{References}
\renewcommand{\bibsection}{}
\bibliography{bibliography.bib} 



%\noindent \hypertarget{tenten}{Корпуса ruTenTen$11$, deTenTen$13$, enTenTen$15$ и BNC  на базе Sketch Engine}:\\\url{https://app.sketchengine.eu} \medskip

%\noindent \hypertarget{glosbe}{Многоязычный онлайн словарь Glosbe}:\;\url{https://ru.glosbe.com} \medskip

%\noindent \hypertarget{ruscorpora}{Национальный корпус русского языка (НКРЯ)}:\;\url{http://ruscorpora.ru} \medskip

%\noindent \hypertarget{gt}{Переводчик Google}:\; \url{https://translate.google.com} \medskip

%\noindent \hypertarget{yt}{Переводчик Yandex}:\; \url{https://translate.yandex.com} \medskip

%\noindent \hypertarget{mlext}{Проект московская лексикотипологическая группа (MLexT)}:\; \url{http://lextyp.org} \medskip

%\noindent \hypertarget{rusvectores}{Семантические модели для русского языка RusVectōrēs}:\;\url{https://rusvectores.org/ru} \medskip

%\noindent Словари и энциклопедии на Академике:\; \url{https://dic.academic.ru} \medskip


\end{document}