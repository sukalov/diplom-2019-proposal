%!TEX TS-program = xelatex
\documentclass[a4paper,12pt]{article}
\usepackage[OT6, T1]{fontenc}
\usepackage[utf-8]{inputenc}
\usepackage[english, russian]{babel}%% загружает пакет многоязыковой вёрстки

\def\changemargin#1#2{\list{}{\rightmargin#2\leftmargin#1}\item[]}
\let\endchangemargin=\endlist 

\usepackage{fontspec}
\setmainfont[
BoldFont=brillb.ttf,
ItalicFont=brilli.ttf,
BoldItalicFont=brillbi.ttf
]{brill.ttf}
	
	%%% Дополнительная работа с математикой
    \usepackage{float}
	\usepackage{amsmath,amsfonts,amssymb,amsthm,mathtools} % AMS
	\usepackage{icomma} % "Умная" запятая: $0,2$ --- число, $0, 2$ --- перечисление
	
	%%% Работа с картинками
	\usepackage{wrapfig} % Обтекание рисунков текстом
	\usepackage{subcaption}
	\usepackage{rotating}
	\usepackage{hhline}
	\usepackage{lscape}
	\usepackage[usenames,dvipsnames,svgnames,table,rgb]{xcolor}%пакет для использования цветов

	\usepackage[inline]{enumitem}
	
	%%% Работа с таблицами
	\usepackage{array,tabularx,tabulary,booktabs} % Дополнительная работа с таблицами
	\usepackage{longtable} % Длинные таблицы
	\usepackage{multirow} % Слияние строк в таблице
	
	\usepackage{multicol} % Несколько колонок
	
	%%% Страница
	\usepackage{extsizes} % Возможность сделать 14-й шрифт
	\usepackage{geometry} % Простой способ задавать поля
	\geometry{top=30mm}
	\geometry{bottom=30mm}
	\geometry{left=30mm}
	\geometry{right=30mm}
	
	%\usepackage{fancyhdr} % Колонтитулы
	% \pagestyle{fancy}
	%\renewcommand{\headrulewidth}{0pt} % Толщина линейки, отчеркивающей верхний колонтитул
	% \lfoot{Нижний левый}
	% \rfoot{Нижний правый}
	% \rhead{Верхний правый}
	% \chead{Верхний в центре}
	% \lhead{Верхний левый}
	% \cfoot{Нижний в центре} % По умолчанию здесь номер страницы
	
	\usepackage{setspace} % Интерлиньяж
	\onehalfspacing % Интерлиньяж 1.5
	%\doublespacing % Интерлиньяж 2
	%\singlespacing % Интерлиньяж 1
	
	\usepackage{lastpage} % Узнать, сколько всего страниц в документе.
	\usepackage{soul} % Модификаторы начертания
	\usepackage{bbding}
	\definecolor{dark-gray}{gray}{0.3}
	\usepackage{hyperref}
	\usepackage[usenames,dvipsnames,svgnames,table,rgb]{xcolor}
	\hypersetup{ % Гиперссылки
	colorlinks=true, % false: ссылки в рамках; true: цветные ссылки
	linkcolor=dark-gray, % внутренние ссылки
	citecolor=black, % на библиографию
	filecolor=black, % на файлы
	urlcolor=ForestGreen % на URL
	}
	
	\usepackage{environ}
	\makeatletter
	\newsavebox{\measure@tikzpicture}
	\NewEnviron{scaletikzpicturetowidth}[1]{%
	\def\tikz@width{#1}%
	\def\tikzscale{1}\begin{lrbox}{\measure@tikzpicture}%
	\BODY
	\end{lrbox}%
	\pgfmathparse{#1/\wd\measure@tikzpicture}%
	\edef\tikzscale{\pgfmathresult}%
	\BODY
	}
	\makeatother
	
    \usepackage{verbatim}
	
	\usepackage{attachfile2}
	\attachfilesetup{appearance=true,
	color=0 0 0
	}
	
	%%% Лингвистические пакеты
	%\usepackage{savetrees} % пакет, который экономит место
	\usepackage{forest} % для рисования деревьев
	\usepackage{vowel} % для рисования трапеций гласных
	\usepackage{natbib}
	\bibliographystyle{ugost2008ns.bst}
	\bibpunct[: ]{[}{]}{;}{a}{}{,}

	\usepackage{philex} % пакет для примеров
    
	\renewcommand{\thesection}{\arabic{section}.}
	\renewcommand{\thesubsection}{\arabic{section}.\arabic{subsection}}
	\setlength{\columnsep}{1.6cm}
	
	\usepackage{sectsty}
	\sectionfont{\normalsize}
	\subsectionfont{\normalsize}
	\usepackage{titlesec}
	\titlespacing*{\section}
	{0pt}{2ex plus 0ex minus .2ex}{0ex plus .2ex}
	\titlespacing*{\subsection}
	{0pt}{2ex plus 0ex minus .2ex}{0ex plus .2ex}
	\newlength{\bibitemsep}\setlength{\bibitemsep}{.2\baselineskip plus .05\baselineskip minus .05\baselineskip}
	\newlength{\bibparskip}\setlength{\bibparskip}{0pt}
	\let\oldthebibliography\thebibliography
	\renewcommand\thebibliography[1]{%
	\oldthebibliography{#1}%
	\setlength{\parskip}{\bibitemsep}%
	\setlength{\itemsep}{\bibparskip}%
	}
\usepackage{tikz}
\usetikzlibrary{arrows,positioning,shapes.geometric}
\usepackage{alltt}
\usepackage{bibunits}
\usepackage{enumitem}
\usepackage{ltablex,booktabs}
\begin{document}
\definecolor{zzttqq}{rgb}{0.26666666666666666,0.26666666666666666,0.26666666666666666}
\definecolor{cqcqcq}{rgb}{0.7529411764705882,0.7529411764705882,0.7529411764705882}
\thispagestyle{empty}
\begin{center}
\noindent 

\textbf{NATIONAL RESEARCH UNIVERSITY HIGHER SCHOOL OF ECONOMICS}\\
\textbf{Faculty of Humanities}\\
\textbf{School of Linguistics}\\
\vfill

\huge{PROJECT PROPOSAL}\\
\large
\LARGE
\textbf{Profiling of the locative participant as the basis of lexical oppositions}\\
\Large
Профилирование локативного участника как основа лексических противопоставлений\\
\vfill
\vfill
\normalsize
\begin{flushright}
Matvey Sokolovskiy, 153\bigskip\\
Linguistic Supervisor:\\
Yury Lander, Ph. D., \\
Associate Professor, School of Linguistics\\
                       
Scientific Supervisor:\\
Tatiana Reznikova, Ph. D., \\
Associate Professor, School of Linguistics\\

\end{flushright}
\vfill
\begin{center}
Moscow --- 2019
\end{center}

\end{center}
\pagebreak

\begin{changemargin}{20mm}{5mm}
\textit{Abstract}\\
%The argument structure of verbs appears to vary in different situations, includingthe same locative participant (LocPar)
The locative participant (LocPar) of the situation can be marked gramatically and syntactically in different ways. In some cases the LocPar is in the position of the direct object (e.g. \textit{to visit X}), in other cases it can be expressed in PP (e.g. \textit{to put something into X}). Moreover, different ways of LocPar marking can refer to the same situation (cf. \textit{to load the truck with smth} vs. \textit{to load smth on the truck}). The aim of this work is to study how the variability in LocPar marking correlates with changes in the verbal root in a typological perspective. In this respect, there are three strategies we can distinguish:
\begin{enumerate*}[label=(\arabic*),itemjoin={\hskip3mm},after=\hskip3mm,before=\hskip3mm]
    \item Structures with a direct object and PP occur with the same verb (\textit{to load} above).
    \item Different structures are realised with verbal roots connected by derivational relations. (cf. Russian verbs \textit{iskat'} and \textit{obyskat'})
    \item Different structures are realised with different verbal roots (cf. \textit{to rob the shop} vs. \textit{to steal smth from the shop}).
\end{enumerate*}
Finally, we hope to observe correlation between the semantics of the verb and the strategy of argument marking, realised with it.\\
\\
\textsc{keywords}:\\
\begin{enumerate*}[label=,itemjoin={\hskip4mm},after=\hskip4mm,before=\hskip4mm]
\item valency;
\item argument structure;
\item locative participant;
\item government pattern;
\item semantic and syntactic arguments;
\item lexical typology;
\item frame-approach;
\item verb derivation

\end{enumerate*}
\end{changemargin}
\normalsize

%{
%  \hypersetup{linkcolor=black}
%  \renewcommand{\contentsname}{Table of contents}
%  \tableofcontents
%}

\section{Introduction} 

A glance thrown at Russian lexemes of 'embezzlement' \textit{krast', grabit', vorovat'} and their closest analogues from English \textit{to rob} and \textit{to steal}, gave birth to the research. We've noticed, that in both English and Russian there are separate lexemes describing the situation with the Locative participant (LocPar) placed in the position of direct object. Hence there might be a root cause at the level of semantics, that suggests profiling the LocPar. The first arising cause is that the LocPar must have some qualities of patient (occupying the position of the direct object in prototypical contexts). The given case is pretty complicated. The LocPar can be replaced with possessor (\textit{to rob the shop} $\leftrightarrow$ \textit{to rob the millionaire}), whose animacy increases the probability of being expressed in the position of DO. But there are contexts, where the LocPar is a direct object and cannot be changed to animate possessor (e.g. Russian verb \textit{podmesti (to sweep)} profiles either Theme or Locative, but never the possessor of the location). So we intend to collect data about a set of ditransitive verbs having a LocPar from several languages\footnote{The certain list of languages is not compiled yet, but for now the first step is to collect data about Russian, English and German languages.}, analyze their argument structures (including those of all possible verbal derivations), and hopefully, find out: \begin{enumerate*}[label=(\roman*),itemjoin={\hskip2mm},after=\hskip2mm,before=\hskip2mm]
\item what entails the existence of several lexemes in one language, describing the same situation,
\item are there situations including more than two semantic arguments, where the position of direct object is default for the LocPar,
\item how can we (if we can) predict the argument structure of a verb in a new language
\end{enumerate*}
The idea to combine the deep lexicographical study of Moscow Semantic School with a typological approach, introduced in \citep{rakhilina2016frame}, looks promising, so we are going to support it and apply deep lexicographic methods to typological data.

\section{Main part}
\subsection{Literature review}
The present paper relates to several areas of linguistic research. 

\par On the one hand it deals with the principles of certain conceptual meaning's lexicalization in typological perspective, which is related to the field of cross-linguistic studies, rapidly growing in recent decades -- lexical typology. It takes roots in the famous book by Brent Berlin and Paul Kay about the typology of color naming \citep{berlin1969basic}. Later several major approaches were evolved:  \begin{enumerate*}[itemjoin={\hskip3mm},after=\hskip3mm,before=\hskip3mm]
    \item The traditional methods are kept through decades in the psycholinguistic studies at the Max Planck Institute in Nijmegen, where the data is collected by recording the speakers’ reactions to extralinguistic stimuli. Its usage can be found in a number of publications by Asifa Majid, the most notable of which are probably the study of cutting and breaking \citep{majid2008cross, majid2007semantic}
    \item The approach introduced by Adrienne Lehrer in  \citep{lehrer1969semantic} is nowadays mainly considered as an outdated. She applied the methods of componential analysis to the analysis of the verbal field of 'cooking'in typological perspective.
    \item The approach, represented by Anna Wierzbicka and Cliff Goddart in \citep{goddard1994semantic} is based on the theory of the universal vocabulary, sufficient to express any meaning in any language.
    \item The last and, for the present, the most promising approach, introduced and developed by Moscow Lexical Typology group and described in details by Ekaterina Rakhilina and Tatiana Reznikova  \citep{rakhilina2016frame} is based on the term of semantic frame, a prototypical meaning, relevant for lexical opposition. Its key point, shared in non-typological studies of Yuri Apresjan \citep{апресян1995избранные}, is that \textit{'lexical meanings can be studied and reconstructed by observing a word’s “surroundings”, or primarily collocation'}.
\end{enumerate*}


%Since the project is going to have a typological orientation, the good way to start is reviewing the methods applicable to a typological research. Those can be found in this book by Martin Haspelmath \citep{haspelmath2001language}. The comparison of languages' lexicalisations shifts us to the 'younger' field of lexical typology, wich takes roots in the famous article of Brent Berlin and Paul Kay about the typology of color naming \citep{berlin1991basic} and later described by Christian Lehmann in \citep{lehmann1990towards} and Maria Koptjevskaja-Tamm in \citep{koptjevskaja2008approaching} anf \citep{krv2015}. But the main focus of the research is argument structure, so the project exists in between lexical and grammatical typologies.

On the other hand, this study considers the problematics of verb's argument structure. Traditionally, the area of verbs' semantic and syntactic valencies is not connected with lexical typology at all. In particular, the argument structure of Russian verbs was deeply analyzed  within the progress of formal lexicography. During the development of the conceptual model "Cмысл $\Leftrightarrow$ Tекст"\footnote{The concept, known as the Meaning-Text theory, was invented and developed by Igor Mel'{\v{c}}uk and a number of Soviet linguists. Actually, MTT is a linguistic framework, for the construction of models of natural language}, the problem was described a lot. Semantic and syntactic arguments of a verb impose restrictions on verb's combinability and it creates a set of problems for a lexicographer. Ideas of solution, (such as including the precise information about word's semantic class, government pattern and combinability into word's definition) and general discussion can be found in \citep[119-156]{апресян1995избранные}, \citep[129-131]{апресян1995избранные2} by Yuri Apresjan and in \citep[134–139]{мельчук1974опыт} by Igor Mel’{\v{c}}uk. A number of more narrow cases, directly related to the present study was discussed among Russian linguists later in \citep{муравенко1998случаях}, \citep{цинман1998модель}. Apart Russian linguists the general concept of the argument structure of a verb was well described by Martin Haspelmath  \citep{haspelmath2004valency}. Lucien Tesnière in \citep{tesniere1959elements} introduces the opposition of semantic arguments (actants) and semantic adjuncts (sirconstants), which will be separately discussed in the paper.
 
\par After all, the discussion of the lexical opposition \textit{to rob -- to steal}, which is considered as the starting point of the current research, appears in \citep[45-48]{goldberg1995constructions}, \citep{thorgren2005transaction} and \citep{van2007role}. The semantic field of 'embezzlement' in Russian contains much more lexicalizations and it is discussed at different angles by Yuri Muratov in his PH.D. thesis \citep{муратовгенетическая} and some earlier articles. But there is no analysis of the field from the argument structure and valency's point of view. 

\subsection{Methods}
The major resources to explore verbs' argument structures are various text corpora and dictionaries' examples. So, the idea is to start with constructing the list of Russian ditransitive (or tritransitive) verbs, having the LocPar among it's arguments. All these verbs can be conditionally\footnote{The division is called conditional because, for example, there is a  Russian verb \textit{загружать (to load)}. It has two different government patterns: object \textit{(what is loaded)} in accusative case + LocPar \textit{(where is loaded)} in PP and vice versa. It cannot be referred to the first group, because the default government pattern is considered to be the first one. And it cannot be referred to the second group either, because there is no derivation. So we would place it between these two given groups.} divided into three groups:
\begin{enumerate}
    \item Verbs, having the LocPar as the direct object originally.
    \item Verbs, which can be derived, thus the LocPar is profiled in the direct object.
    \item Verbs, which do not allow profiling of the LocPar.
\end{enumerate}
After getting the list it is possible to find those verbs' analogues in other languages. The most precise way to do it is to use parallel corpora provided by Russian National Corpus (\small{ \url{http://ruscorpora.ru/}}) \normalsize and bilingual dictionaries. Parallel corpora give us the opportunity to explore, how the same situation's lexical expression is arranged in several languages. Combining the information about verbs from all discussed languages we will compile the list of semantic frames\footnote{Basic polyvalent sittuations including the LocPar.}. Then it can be sorted by the position of the LocPar, which means from frames, which often have a basic lexeme with LocPar in the position of direct object to those, which cannot express the LocPar in the position of direct object at all (or almost never can). 

\subsection{Anticipated results}

It is considered, that the position of direct object is prototypically occupied by the patient. However the insances of placing there the LocPar instead of the patient are pretty common. We hope to find the correlation between the syntactical position of the LocPar and its role in the situation generally. 

\section{Conclusion}
Once again, we anticipate that the research will prove the  dependency between the importance of the LocPar in the frame-situation and probability of its placing to the position of direct object, 

%The locative participant of a situation was found 

\section{References}
\renewcommand{\bibsection}{}
\bibliography{bibliography.bib} 



%\noindent \hypertarget{tenten}{Корпуса ruTenTen$11$, deTenTen$13$, enTenTen$15$ и BNC  на базе Sketch Engine}:\\\url{https://app.sketchengine.eu} \medskip

%\noindent \hypertarget{glosbe}{Многоязычный онлайн словарь Glosbe}:\;\url{https://ru.glosbe.com} \medskip

%\noindent \hypertarget{ruscorpora}{Национальный корпус русского языка (НКРЯ)}:\;\url{http://ruscorpora.ru} \medskip

%\noindent \hypertarget{gt}{Переводчик Google}:\; \url{https://translate.google.com} \medskip

%\noindent \hypertarget{yt}{Переводчик Yandex}:\; \url{https://translate.yandex.com} \medskip

%\noindent \hypertarget{mlext}{Проект московская лексикотипологическая группа (MLexT)}:\; \url{http://lextyp.org} \medskip

%\noindent \hypertarget{rusvectores}{Семантические модели для русского языка RusVectōrēs}:\;\url{https://rusvectores.org/ru} \medskip

%\noindent Словари и энциклопедии на Академике:\; \url{https://dic.academic.ru} \medskip


\end{document}